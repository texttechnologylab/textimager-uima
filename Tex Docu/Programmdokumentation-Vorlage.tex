\documentclass[pagesize=auto, titlepage=on]{scrartcl}

\usepackage[utf8]{inputenc}
\usepackage[T1]{fontenc}
\usepackage{lmodern}
\usepackage[ngerman]{babel}
\usepackage[babel]{csquotes}
\usepackage[style=authoryear]{biblatex}
\addbibresource{Beispiel.bib}
\usepackage{graphicx}
\usepackage{xcolor}
\usepackage{booktabs}
\usepackage{listings}
\lstset{basicstyle=\ttfamily\small, numbers=left,
  numberstyle=\scriptsize\color{black!70}\sffamily, 
  numbersep=3pt 
}
\usepackage{tikz}
\usetikzlibrary{arrows}
\usetikzlibrary{shapes.geometric}
\usetikzlibrary{positioning}
\tikzset{
  Entscheidung/.style={diamond, draw, fill=gray!40, text
    width=2cm, align=center},
  Element/.style={rectangle, draw, text width=2.5cm, align=center},
  Nutzer/.style={ellipse, draw, minimum height=2em, text width=1.6cm,
    align=center, text=white, fill=black!70} 
}
    

\begin{document}
\titlehead{Seminar XY \hfill Goethe-Universtität Frankfurt am Main}
\subject{Typisierung der Arbeit}
\author{Vorname Name \\ Matrikelnummer}
\title{Titel der Arbeit}
\subtitle{Ggf. Untertitel}
\publishers{Betreuer}

\maketitle

\tableofcontents

\section{Einleitung}
\label{sec:einleitung}
Dieses Dokument bietet eine Vorlage zur Strukturierung einer Programm- oder Projektdokumentation.
Der Umfang ergibt sich aus der adäquaten Darstellung der Projektarbeit und variiert je nach Thematik.
Als ungefährer Richtwert kann eine Seitenzahl von 12--16 Seiten gelten.
Die Abgabe sollte als PDF erfolgen und auch die Quelldokumente (\LaTeX, Open Office, \ldots) beinhalten.
Zur Abfassung der Dokumentation wird \LaTeX empfohlen, es können aber auch andere Programme genutzt werden.
Zusätzlich enthält die Abgabe die dokumentierten Programmquelltexte, eine ausführbare Version und erstelle Ressourcen
soweit dies im Rahmen der Aufgabenstellung zutreffend ist.

\subsection{Problemstellung}
\label{sec:problemstellung}
In diesem Abschnitt folgt eine möglichst präzise Formulierung der Aufgabenstellung in eigenen Worten.
Dies dient nicht zuletzt auch der eigenen Reflexion und Kontrolle über das Verständnis der Arbeit.

\subsection{Motivation}
\label{sec:motivation}
Hier sollte kurz dargestellt werden, was der Mehrwert einer Lösung der gegebenen Problemstellung ist.
Das könnte beispielsweise eine neue Ressource sein, die als Grundlage für zukünftige Analysen dienen kann.
Bei der Durchführung von Analysen oder Evaluationen stünde dagegen der daraus resultierende Erkenntnisgewinn im Vordergrund.
Es kann sich aber auch primär um ein Programm handeln, dass eine bestimmte Funktionalität bietet, wie z.B. eine automatische Klassifikation oder eine Konversion von Daten.

\section{Lösungsweg}
\label{sec:ansatz}

\subsection{Ausgangslage}
\label{sec:ausgangslage}
Zunächst sollte die Ausgangslage beschrieben werden:
Welche Ressourcen und Programme können oder sollen konkret zur Lösung der Aufgabenstellung herangezogen werden?
Dabei sollte klar ersichtlich sein, aus welchen Quellen die Ressourcen, Algorithmen oder Konzepte stammen.

Beispiel: \enquote{Als Lösungsansatz für das Problem (vgl. Abschnitt \ref{sec:problemstellung})
folge ich \textcite[S.~214]{Manning:Schuetze:1999}. \ldots.}

\subsection{Vorgehensweise}
Diese Sektion greift die eingangs dargelegte Problemstellung auf und beschreibt den gewählten Lösungsweg.
Die Vorgehensweise sollte hier verständlich und nachvollziehbar beschrieben werden.
Dazu ist es hilfreich, wenn sich der Leser an einem Ablauf- oder Flussdiagramm orientieren kann (vgl. Beispiel in Abbildung \ref{fig:flowchart}).
Die Beschreibung soll den Lösungsweg verständlich machen, ohne sich dabei in Details zu verlieren.
Besondere Aspekte, etwa ein angewandter ``Kniff'' können hier dennoch erwähnt werden.

%Das logische Modell meines Lösungsansatzes ist in Abbildung
%\ref{fig:flowchart} dargestellt. Sowohl der Benutzer (\textit{User})
%als auch das System können den Modellprozess starten. Nach der
%Initialisierung werden $n$ Kandidatenmodelle ausgewählt und
%evaluiert. Wenn ein Kandidat besser abschneidet als das bisherige
%beste Modell, wird der Prozess mit dem neuen besten Modell
%aktualisiert (\textit{Update}). Andernfalls haben wir ein bestes
%Modell gefunden und erreichen den terminalen Zustand (\textit{Stop}).

\begin{figure}[tb]
  \centering
  \begin{tikzpicture}[font=\footnotesize]
    % Positioniere Knoten:
    \node [Element] (init) {initialize model};
    \node [Nutzer, left=of init] (expert) {expert};
    \node [Nutzer, right=of init] (system) {system};
    \node [Element, below=of init] (identify) {identify candidate models};
    \node [Element, below=of identify] (evaluate) {evaluate candidate models};
    \node [Element, left=of evaluate, node distance=3cm] (update) {update model};
    \node [Entscheidung, below=of evaluate] (decide) {is best candidate better?};
    \node [Element, below=of decide, node distance=3cm] (stop) {stop};
    % Zeichne Kanten:
    \begin{scope}[->, >=stealth]
      \draw (init) -- (identify);
      \draw (identify) -- (evaluate);
      \draw (evaluate) -- (decide);
      \draw (decide) -| node [near start, above] {yes} (update);
      \draw (update) |- (identify);
      \draw (decide) -- node [midway, right] {no}(stop);
      \draw [dashed] (expert) -- (init);
      \draw [dashed] (system) -- (init);
      \draw [dashed] (system) |- (evaluate);
    \end{scope}
  \end{tikzpicture}
  \caption{Flussdiagramm des Lösungsansatzes}
  \label{fig:flowchart}
\end{figure}


\section{Implementation}
\label{sec:implementation}
\subsection{Architektur}
Zu Beginn sollte die Beschreibung der Architektur stehen: Welche Komponenten gibt es und wie spielen diese zusammen?
Zum besseren Verständnis ist bei nicht-trivialen System ein Diagramm sinnvoll an dem sich der Leser orientieren kann.
Besteht das Programm nicht nur aus einem überschaubaren Algorithmus sondern aus mehreren Klassen, sollten diese anhand eines UML Klassendiagramm beschrieben werden.

\subsection{Dokumentation der Quelltexte}
\label{sec:quelltextdokumentation}
Ein ggf. implementiertes Programm muss im Quelltext angemessen dokumentiert sein.
Für Java Programme ist der etablierte Standard JavaDoc\footnote{\url{http://de.wikipedia.org/wiki/Javadoc}} zu verwenden.
Dokumentiert werden sollten zumindest alle Klassen und Methoden sowie deren Parameter.
Bei nicht trivialen Methoden ist auch eine Kommentierung einzelner Code-Blöcke für das Verständnis hilfreich.
Diese Dokumentation ist als PDF Dokument oder Html-Struktur Teil der (elektronischen) Abgabe. 

Falls innerhalb dieses Dokuments Quelltextausschnitte zitiert werden sollen, bietet z.B. Latex spezielle Konstrukte. Beispiel:

\begin{lstlisting}[language=Java]
while (!list.isEmpty()) {
  process(list.remove(0));
}
\end{lstlisting}

\subsection{Anwenderdokumentation}
\label{sec:anwenderdokumentation}
Die Anwenderdokumentation beschreibt den Aufruf sowie die Benutzung des Programms.
Dazu zählen auch ggf. erforderliche Installationsschritte und die Systemvoraussetzungen (Betriebssystem, installierte Bibliotheken etc.).

\section{Analyse/Evaluation}
\label{sec:evaluation}
In dieser Sektion können, falls dies für die Aufgabenstellung zutreffend ist, durchgeführte Analysen oder Evaluationen und deren Interpretation beschrieben werden.
Im Falle einer Kategorisierungsaufgabe könnte hier zum Beispiel eine Beschreibung des Testkorpus, der Parameter sowie der erzielten Ergebnisse stehen.

\section{Zusammenfassung}
\label{sec:zusammenfassung}
Hier wird kurz noch einmal zusammenfassend beschrieben, worum es in der vorliegenden Arbeit geht und was die Ergebnisse sind.
Auch sollte hier auf offene Punkte hingewiesen werden und ein Ausblick auf mögliche weitere Schritte gegeben werden.

\printbibliography


\section*{Erklärung}

Hiermit erkläre ich, dass ich die vorliegende Seminararbeit
selbstständig verfasst und keine anderen als die angegebenen
Hilfsmittel benutzt habe. Die Stellen der Seminararbeit, die anderen
Quellen im Wortlaut oder dem Sinn nach entnommen wurden, sind durch
Angaben der Herkunft kenntlich gemacht. Dies gilt auch für
Zeichnungen, Skizzen, bildliche Darstellungen sowie für Quellen aus
dem Internet.\\[2\baselineskip]
Ort, \today \\[5\baselineskip]
Vorname Name


\end{document}
